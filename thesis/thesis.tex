% Opcje klasy 'iithesis' opisane sa w komentarzach w pliku klasy. Za ich pomoca
% ustawia sie przede wszystkim jezyk i rodzaj (lic/inz/mgr) pracy, oraz czy na
% drugiej stronie pracy ma byc skladany wzor oswiadczenia o autorskim wykonaniu.
\documentclass[declaration,shortabstract,inz]{iithesis}

\usepackage[utf8]{inputenc}

%%%%% DANE DO STRONY TYTUŁOWEJ
% Niezaleznie od jezyka pracy wybranego w opcjach klasy, tytul i streszczenie
% pracy nalezy podac zarowno w jezyku polskim, jak i angielskim.
% Pamietaj o madrym (zgodnym z logicznym rozbiorem zdania oraz estetyka) recznym
% zlamaniu wierszy w temacie pracy, zwlaszcza tego w jezyku pracy. Uzyj do tego
% polecenia \fmlinebreak.
\polishtitle    {Badanie gier kooperacyjnych z niepełną\fmlinebreak informacją na przykładzie gry Hanabi}
\englishtitle   {A study on cooperative games with incomplete\fmlinebreak information based on the game of Hanabi}
\polishabstract {\ldots}
\englishabstract{\ldots}
% w pracach wielu autorow nazwiska mozna oddzielic poleceniem \and
\author         {Wojciech Jarząbek \and
				Jacek Leja}
% w przypadku kilku promotorow, lub koniecznosci podania ich afiliacji, linie
% w ponizszym poleceniu mozna zlamac poleceniem \fmlinebreak
\advisor        {dr Paweł Rychlikowski}
%\date          {}                     % Data zlozenia pracy
% Dane do oswiadczenia o autorskim wykonaniu
%\transcriptnum {}                     % Numer indeksu
%\advisorgen    {dr. Pawła Rychlikowskiego} % Nazwisko promotora w dopelniaczu
%%%%%

%%%%% WLASNE DODATKOWE PAKIETY
%
%\usepackage{graphicx,listings,amsmath,amssymb,amsthm,amsfonts,tikz}
%
%%%%% WŁASNE DEFINICJE I POLECENIA
%
%\theoremstyle{definition} \newtheorem{definition}{Definition}[chapter]
%\theoremstyle{remark} \newtheorem{remark}[definition]{Observation}
%\theoremstyle{plain} \newtheorem{theorem}[definition]{Theorem}
%\theoremstyle{plain} \newtheorem{lemma}[definition]{Lemma}
%\renewcommand \qedsymbol {\ensuremath{\square}}
% ...
%%%%%

\begin{document}

%%%%% POCZĄTEK ZASADNICZEGO TEKSTU PRACY

\chapter{Wprowadzenie}

Gry planszowe to forma rozrywki, która towarzyszy człowiekowi od~tysięcy lat. Były one popularne już za~czasów antycznych, czego dowodzi chociażby malowidło z~3300~r.~p.n.e, pochodzące z~grobowca Merknery, na~którym ukazano rozgrywkę Seneta. Przykładem może być także Królewska~Gra~z~Ur, której egzemplarze odnaleziono w~trakcie badań nad~starożytną Mezopotamią. Choć zostały one~w~dzisiejszych czasach w~znacznej mierze zapomniane, nie~sposób nie~wspomnieć o~innych grach z~podobnego okresu, takich~jak~warcaby czy~Go, a~także nieco młodszych szachach, które wciąż cieszą~się ogromną i~niesłabnącą popularnością.

Każdą z~tych gier łączy aspekt rywalizacji: pod~koniec rozgrywki jednoznacznie wyszczególnia~się jednego lub~więcej graczy, których określamy mianem zwycięzców, zaś~reszta -~przegrywa. Zgoła inne podejście prezentują gry kooperacyjne, w~których zadaniem nie~jest pokonanie innych uczestników zabawy, a~osiągnięcie wspólnego celu, który gwarantuje wygraną. Można powiedzieć, że~przeciwnikiem graczy jest w~tym przypadku sama gra, która swoją konstrukcją skłania do~współpracy. Co~może zaskakiwać, pierwsze gry tego typu powstały dopiero w~drugiej połowie XX~wieku i~początkowo miały wyłacznie charakter edukacyjny. Wraz z~popularyzacją tzw. "planszówek" gry kooperacyjne w~znacznym stopniu zyskały na~popularności, a~ich forma wyewoluowała w~kierunku zabawy stawiającej na~aspekty towarzyskie, które w~znacznym stopniu ograniczają, lub~wręcz odrzucają współzawodnictwo. Przykładami takiego podejścia mogą~być Pandemic, Martwa Zima, a~także Hanabi, na~którym skupia~się nasza praca.

Hanabi (jap. fajerwerki) to~w~pełni kooperacyjna gra planszowa, która w~2013 roku wygrała prestiżową nagrodę Spiel~des~Jahres. Gracze wcielają~się w~niej w pracowników fabryki fajerwerków, w~której to~pomieszane zostały rodzaje prochu. Celem jest złożenie w~odpowiedniej kolejności możliwie jak~największej ilości sztucznych ogni, które gracze otrzymują poprzez dobieranie kart z~potasowanej talii. Uczestnicy rozgrywki widzą karty, które są~w~posiadaniu innych graczy, lecz~nie~mogą przypatrywać~się~tym, którymi sami dysponują. Dodatkowo, komunikacja odnosząca~się do~treści kart podlega restrykcyjnym zasadom i~jest w~znacznym stopniu ograniczona, co~w~oczywisty sposób czyni rozgrywkę niebanalną. Jakie strategie należy zatem zastosować, by~wygrać? Jak można przełożyć je~na~świat elektronicznej rozrywki?

W~teorii gier istnieje pojęcie perfekcyjnego zagrania, czyli pojedynczego ruchu zależnego od~aktualnego etapu gry, który prowadzi do~stanu rozgrywki maksymalizującego oczekiwany wynik, niezależnie od~ruchów, które mogą w~odpowiedzi wykonać inni gracze. Perfekcyjne zagrania są~podstawą optymalnego planu działania, który minimalizuje możliwe straty ponoszone w~trakcie rozgrywania danej partii. Niestety, tak~silna strategia w~przypadku złożonych gier jest nieprawdopodobnie trudna do~uzyskania ze~względu na~niewyobrażalną rozpiętość drzewa możliwych do~uzyskania stanów rozgrywki. W~praktyce używa~się algorytmów heurystycznych, regułowych, opartych na~technikach uczących, nadużywających zasad gry lub~siłowych. Przykładowo, słynny komputer Deep Blue, który w~maju 1997 roku pokonał ówczesnego mistrza świata w~szachach, Garrego Kasparova, nie~posiadał optymalnej strategii. Używał on~w~zamian metody siłowej, wspomaganej algorytmem przeszukującym alfa-beta, rozpatrując wszystkie możliwe zagrania i~wybierając~te, które dawały mu~największą przewagę lokalną. Takie podejście było możliwe z~racji na ogromną moc superkomputera, który potrafił rozpatrywać 200~milionów ruchów na~sekundę.

Stworzenie sztucznej inteligencji do~Hanabi to~zadanie, które wymaga pokonania trudności niespotykanych w~innych grach. Jest to~następstwo kilku czynników: niepełnej informacji, losowości dobieranych kart, a~także ograniczonych zasobów, m.in. w~postaci podpowiedzi dla~innych graczy. Agenci muszą sobie ufać, gdyż gracz, który nie~chce współpracować, może w~kilku ruchach doprowadzić do~przegranej całej grupy. Ważne jest, by~nie~marnować zasobów, a~zatem sztuczna inteligencja musi być odpowiednio skoordynowana z~innymi graczami. Ponadto, znikoma ilość kart w~talii nie~pozwala na~wygraną w sytuacji, w~której zagrywane są~wyłącznie karty z~pełną informacją. Oznacza to~zatem, że~agenci muszą posiadać protokół komunikacji, który pozwala im~na~przekazywanie w~obrębie zasad gry dodatkowych, implicytnych informacji, rozumianych przez pozostałych jej~uczestników.

Niniejsza praca ma~na~celu przedstawienie technik tworzenia agentów sztucznej inteligencji grających w~Hanabi, którzy wykonują możliwie jak~najbardziej intuicyjne ruchy na~tyle szybko, by~umożliwić komfortową rozgrywkę z~ludzkimi graczami na~zwykłych komputerach.

\chapter{Zasady gry}



%%%%% BIBLIOGRAFIA

%\begin{thebibliography}{1}
%\bibitem{example} \ldots
%\end{thebibliography}

\end{document}
