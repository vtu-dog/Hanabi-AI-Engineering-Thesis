\documentclass[declaration,shortabstract,inz]{iithesis}

\polishtitle    {Badanie gier kooperacyjnych z niepełną\fmlinebreak informacją na przykładzie gry Hanabi}
\englishtitle   {A study on cooperative games with incomplete\fmlinebreak information based on the game of Hanabi}
\polishabstract {\ldots}
\englishabstract{\ldots}
\author         {Wojciech Jarząbek \and
				Jacek Leja}
\advisor        {dr Paweł Rychlikowski}
\date          {\today}                     % Data zlozenia pracy
% Dane do oswiadczenia o autorskim wykonaniu
%\transcriptnum {}                     % Numer indeksu
%\advisorgen    {dr. Pawła Rychlikowskiego} % Nazwisko promotora w dopelniaczu
%%%%%


\usepackage[utf8]{inputenc}
\usepackage{graphicx}
\usepackage{caption}
\usepackage{float}

\graphicspath{{img/}}
\let\cleardoublepage=\clearpage

\begin{document}

\chapter{Wprowadzenie}

\section{Czym jest Hanabi?}

Gry planszowe to forma rozrywki, która towarzyszy człowiekowi od~tysięcy lat. Były one popularne już za~czasów antycznych, czego dowodzi chociażby malowidło z~3300~r.~p.n.e, pochodzące z~grobowca Merknery, na~którym ukazano rozgrywkę Seneta. Przykładem może być także Królewska~Gra~z~Ur, której egzemplarze odnaleziono w~trakcie badań nad~starożytną Mezopotamią. Choć gry te zostały w~dzisiejszych czasach w~znacznej mierze zapomniane, nie~sposób nie~wspomnieć o~innych z~podobnego okresu, takich~jak~warcaby czy~Go, a~także o~nieco młodszych szachach, które wciąż cieszą~się ogromną i~niesłabnącą popularnością.

Każdą z~tych gier łączy aspekt rywalizacji: pod~koniec rozgrywki jednoznacznie wyszczególnia~się jednego lub~więcej graczy, których określamy mianem zwycięzców, zaś~reszta -~przegrywa. Inne podejście prezentują gry kooperacyjne, gdzie zadaniem nie~jest pokonanie innych uczestników zabawy, a~osiągnięcie wspólnego celu, który gwarantuje wygraną. Można powiedzieć, że~przeciwnikiem graczy jest w~tym przypadku sama gra, która swoją konstrukcją skłania do~współpracy. Pierwsze gry tego typu powstały dopiero w~drugiej połowie XX~wieku i~początkowo miały wyłacznie charakter edukacyjny. Wraz z~popularyzacją tzw. ``planszówek'', gry kooperacyjne w~znacznym stopniu zyskały na~popularności, a~ich forma wyewoluowała w~kierunku zabawy kładącej nacisk na~aspekty towarzyskie, które ograniczają lub~wręcz odrzucają współzawodnictwo. Przykładami takiego podejścia mogą~być Pandemic, Martwa Zima, a~także Hanabi.

Hanabi (jap. fajerwerki) to~w~pełni kooperacyjna gra planszowa, która w~2013 roku wygrała prestiżową nagrodę Spiel~des~Jahres. Gracze wcielają~się w~niej w pracowników fabryki fajerwerków, w~której omyłkowo zostały pomieszane ze~sobą różne rodzaje prochu. Celem jest złożenie w~odpowiedniej kolejności możliwie jak~największej ilości sztucznych ogni, które gracze otrzymują poprzez dobieranie kart z~potasowanej talii. Uczestnicy rozgrywki widzą karty, które są~w~posiadaniu innych graczy, lecz~nie~mogą przypatrywać~się~tym, którymi sami dysponują. Dodatkowo, komunikacja odnosząca~się do~treści kart podlega restrykcyjnym zasadom i~jest w~znacznym stopniu ograniczona, co~czyni rozgrywkę nietrywialną. Jakie strategie należy zatem zastosować, by~wygrać? Jak można przełożyć je~na~świat algorytmów?

\section{Hanabi a sztuczna inteligencja}

W~teorii gier istnieje pojęcie perfekcyjnego zagrania, czyli pojedynczego ruchu zależnego od~aktualnego etapu gry, prowadzącego do~stanu rozgrywki maksymalizującego oczekiwany wynik, niezależnie od~ruchów, które mogą w~odpowiedzi wykonać inni gracze. Perfekcyjne zagrania są~podstawą optymalnego planu działania, minimalizującego możliwe straty ponoszone w~trakcie rozgrywania danej partii. Niestety, tak~silna strategia -~w~przypadku złożonych gier -~jest nieprawdopodobnie trudna do~uzyskania ze~względu na~ogromną rozpiętość drzewa możliwych do~uzyskania stanów rozgrywki. W~praktyce używa~się algorytmów: heurystycznych, regułowych, opartych na~technikach uczących, nadużywających zasad gry lub~siłowych. Przykładowo, słynny komputer Deep Blue, który w~maju 1997 roku pokonał ówczesnego mistrza świata w~szachach, Garrego Kasparova, nie~posiadał optymalnej strategii. Używał on~w~zamian metody siłowej, wspomaganej algorytmem przeszukującym alfa-beta, rozpatrując wszystkie możliwe zagrania i~wybierając~te, które dawały mu~największą przewagę lokalną. Takie podejście było możliwe z~racji na ogromną moc superkomputera, który potrafił rozpatrywać 200~milionów ruchów na~sekundę.

Stworzenie agentów sztucznej inteligencji do~Hanabi to~zadanie, które wymaga pokonania trudności niespotykanych w~innych grach. Jest to~następstwo kilku czynników: niepełnej informacji, losowości dobieranych kart, a~także ograniczonych zasobów, m.in. w~postaci podpowiedzi dla~innych graczy. Agenci muszą wspólpracować, gdyż gracz, który odmawia~kooperacji, może w~kilku ruchach doprowadzić do~przegranej całej grupy. Ważne jest, by~nie~marnować zasobów, a~zatem sztuczna inteligencja musi być odpowiednio skoordynowana. Ponadto, znikoma ilość kart w~talii nie~pozwala na~zbyt długą rozgrywkę -~oznacza to~zatem, że~aby zdążyć z~wygraną, agenci muszą posiadać protokół komunikacji, który dopuszcza przekazywanie w~obrębie zasad gry dodatkowych, implicytnych informacji, rozumianych przez pozostałych jej~uczestników.

Niniejsza praca ma~na~celu zbadanie Hanabi jako gry kooperacyjnej z~niepełną informacją. Będziemy analizować techniki tworzenia agentów sztucznej inteligencji grających w~Hanabi, którzy wykonują możliwie najbardziej efektywne i~zrozumiałe dla ludzi ruchy na~tyle szybko, by~umożliwić komfortową rozgrywkę z~człowiekiem na~zwykłych komputerach.


\chapter{Reguły gry}

%\renewcommand{\thefigure}{1}
\begin{figure}[ht!]
	\centering
	\captionsetup{format=hang}
	\includegraphics[width=\textwidth,height=\textheight,keepaspectratio]{gui.png}
	\caption[Caption]{Interfejs graficzny do gry Hanabi, utworzony na potrzeby projektu (twórca grafik: Jakub Podwysocki)}
	\label{fig:gui}
\end{figure}

\section{Wyjaśnienie zasad}

Celem gry jest zdobycie możliwie największej ilości punktów poprzez poprawne zagrywanie kart. Maksymalna ilość możliwych do~uzyskania punktów wynosi dwadzieścia pięć. Po~zakończeniu gry ilość uzyskanych punktów oblicza~się poprzez zsumowanie wartości najwyższych kart z~każdego ze~stosów odpowiedniego koloru.

Talia do~gry składa~się z~pięćdziesięciu kart. Każda karta jest~oznaczona jednym z~pięciu kolorów (czerwony, żółty, niebieski, biały, zielony) oraz jedną z~wartości z~zakresu od~1~do~5. Dla każdego koloru istnieją po~trzy karty o~numerze~1, po~dwie karty o~numerach 2,~3~i~4, a~także po~jednej karcie o~numerze~5.

Na początku gry talia jest tasowana. Gracze rozpoczynają rozgrywkę z~ośmioma żetonami podpowiedzi i~trzema żetonami życia. Żetony te są wspólne dla wszystkich uczestników rozgrywki. Jeżeli graczy jest dwóch lub~trzech, każdy z~nich dobiera po~pięć zakrytych kart. Jeżeli jest ich~czterech lub~pięciu, dobierają po~cztery zakryte karty. Następnie gracze po kolei wykonują swoje ruchy. Ruchu nie~można pominąć. Ruch to~wykonanie jednej z~trzech dostępnych akcji:
\begin{enumerate}
	\item Zagranie karty:

	Gracz deklaruje chęć zagrania karty, wybiera zakrytą ze~swojej ręki, a~następnie wykłada ją~na~planszę w~pozycji odkrytej. Zagranie może być~poprawne lub~niepoprawne. Karty muszą być~zagrywane w~kolejności rosnącej, zaczynając od~jedynki, inaczej zagranie uważa~się za~niepoprawne. Przykładowo, jeśli na~planszy~nie ma~żadnych kart, można poprawnie zagrać tylko te, które są oznaczone numerem~1. Jeżeli na~planszy znajdują~się wyłącznie jedna niebieska karta o~numerze~1 i~stos żółtych kart, spośród których największą wartość ma~karta o~numerze~4, można poprawnie zagrać niebieską kartę o~numerze~2, żółtą kartę o~numerze~5 lub dowolną kartę innego koloru o~numerze~1. Jeżeli karta została zagrana poprawnie, jest ona dodawana do~stosu o~odpowiednim kolorze lub~też~rozpoczyna stos swojego koloru, jeżeli jest to~karta o~numerze~1. Dodatkowo, jeżeli zagrana karta ma~numer~5, gracze otrzymują jeden żeton podpowiedzi (chyba, że~mają ich już osiem -~wtedy zagranie nie~ma~żadnego dodatkowego efektu). Jeżeli karta została zagrana niepoprawnie, jest ona~usuwana z~gry i~nie~jest dodawana do~żadnego ze~stosów, a~gracze tracą jeden z~żetonów życia. Po~rozpatrzeniu efektów akcji gracz dobiera zakrytą kartę z~talii (jeżeli nie~jest ona~pusta).
	
	\item Odrzucenie karty:
 
	Gracz deklaruje chęć odrzucenia karty, wybiera zakrytą ze~swojej ręki, a~następnie wykłada ją~na~planszę w~pozycji odkrytej. Karta ta~jest usuwana z~gry, bez dokładania jej~do~któregokolwiek ze~stosów, a~gracze otrzymują jeden żeton podpowiedzi (chyba, że~mają ich już osiem -~wtedy odrzucenie karty nie~ma~żadnego dodatkowego efektu). Po~rozpatrzeniu efektów akcji gracz dobiera zakrytą kartę z~talii (jeżeli nie~jest ona~pusta).

	\item Udzielenie podpowiedzi innemu graczowi:

	Gracz usuwa jeden z~żetonów podpowiedzi, po~czym wybiera innego uczestnika rozgrywki oraz jeden z~dwóch rodzajów informacji, których chce mu~udzielić: może wskazać wszystkie jego karty o~wybranym kolorze lub~wszystkie jego karty o~wybranym numerze. Akcji tej nie~można wykonać, jeśli w~grze~nie ma~żadnych żetonów podpowiedzi, gdyż wiązałoby~się to~z~koniecznością usunięcia żetonu podpowiedzi, który~nie istnieje. Udzielanie graczom podpowiedzi dotyczących kart w~jakikolwiek inny sposób jest zabronione.
\end{enumerate}

Jeżeli któryś z graczy dobierze ostatnią kartę z~talii, każdy uczestnik rozgrywa jedną dodatkową turę (wraz z~graczem, który dobrał ostatnią kartę), po~czym gra~się kończy.

Gra natychmiast kończy~się, gdy~zostanie utracony ostatni żeton życia lub gdy~wszystkie stosy odpowiednich kolorów zostaną skompletowane (czyli na~każdy z~nich poprawnie położono kartę o~numerze~5).

\section{Dodatkowe obserwacje}

Partie na~dwóch graczy cechują~się bardzo wysokim stopniem trudności. Uczestnicy rozgrywki znacznie częściej muszą radzić sobie ze~skomplikowanymi sytuacjami, które wynikają z~niekorzystnych rozdań. Fakt ten jest spotęgowany przez znikomą ilość dostępnych informacji (każdy z~graczy widzi naraz tylko pięć kart), a~także łatwość wpadnięcia w~cykl, w~którym jeden z~graczy udziela podpowiedzi, a~drugi zagrywa lub odrzuca karty.

Z powodu losowej natury gry, niektórych rozdań nie~da~się wygrać z~maksymalną ilością punktów. Najprostsza taka sytuacja ma~miejsce, gdy w~rozgrywce na~dwóch graczy jeden z~nich dobierze same karty o~numerze~5, drugi dobierze wyłącznie karty o~numerze~4, a~na~górze talii znajdują~się pozostałe karty o~numerze~4. Aby uzyskać dostęp do~kart o~innych numerach, gracze muszą odrzucić (lub niepoprawnie zagrać) co~najmniej sześć kart. Z~zasady szufladkowej Dirchleta można wywnioskować, że~wszystkie kopie co~najmniej jednej z~kart danego rodzaju zostaną bezpowrotnie usunięte z~gry, bez umieszczania ich na~stosie, co~uniemożliwia wygraną.

Ze~względu na~ograniczony rozmiar talii, zagrywanie wyłącznie tych kart, o~których posiada~się komplet informacji, jest wysoce nieefektywne. Po~rozdaniu kart talia zawiera od~30~do~40 kart, zależnie od~liczby graczy. Aby uzyskać najwyższy wynik, należy zagrać aż~25~kart, a~zagranie każdej z~nich oznacza zmniejszenie rozmiaru talii o~jeden. Oznacza to (zakładając, że~gracze próbują uzyskać 25~punktów), że~można wykonać maksymalnie od~10~do~17 ruchów, w~których odrzuca~się kartę, wliczając w~to~tury po~opróżnieniu talii. Po~doliczeniu początkowych 8~żetonów daje to~maksymalną ilość 25~podpowiedzi. Oznacza to, że~w~grze na~dwie osoby każda podpowiedź musi jednoznacznie ujawniać średnio po~jednej karcie, lecz każda z~nich potrzebuje dwóch podpowiedzi różnego rodzaju, by~uzyskać pełną informację. Przy pięciu graczach każda podpowiedź musi średnio ujawniać już nie~jedną, a~prawie półtorej karty.

Ponadto, jeżeli chcemy odrzucać wyłącznie karty, które można bezpiecznie usunąć z~gry, inni gracze muszą nam zakomunikować ich brak przydatności poprzez odpowiednie podpowiedzi (lub ich brak, co~jest w znacznym stopniu utrudnione przez konieczność udzielania pełnej informacji o~zawartości rąk współuczestników). Karty bezużyteczne mogą, lecz nie~muszą zostać ujawnione w~drodze przypadku, podczas ujawniania innych kart. W~rezultacie najczęściej przyjmowaną konwencją wśród graczy jest odrzucanie najstarszej karty w~ręce: jeżeli żaden z~uczestników rozgrywki nie~ostrzega przed usunięciem jej z~gry, istnieje wysokie prawdopodobieństwo, że~jest ona nieprzydatna.

\chapter{Strategie sztucznej inteligencji}

\section{Podejście regułowe}

Podczas realnej rogrywki Hanabi gracze nie dysponują komputerem, który pomagałby im~w~wykonywaniu ruchów poprzez dokonywanie odpowiednich obliczeń. Zamiast tego korzystają oni z~wiedzy nabytej w~trakcie rozegranych już partii, modyfikując swoje strategie i~rozszerzając je~o~kolejne elementy, aż~do~osiągnięcia zadowalającego ich poziomu wiedzy o~grze. Proces ten w~naturalny sposób prowadzi do~wykształcenia konwencji (takich jak: ``należy zawsze odrzucać najstarszą kartę w~ręce''), a~także do~opracowania systemu reguł, pomocnych w~uzyskiwaniu wysokich wyników (przykładowo: ``jeżeli ktoś chce odrzucić ważną kartę, należy go~powstrzymać poprzez udzielenie odpowiedniej podpowiedzi''). Zasady te~można przełożyć na~algorytmy, co~prowadzi do~utworzenia systemów regułowych, znanych także jako eksperckie.

Algorytmy regułowe to~programy, które poprzez procesy decyzyjne naśladują wybory, których w~danych sytuacjach mógłby dokonać człowiek. Są~one najczęściej deterministyczne, co~jest cechą szczególnie ważną w~środowiskach, które wymagają koordynacji i~przewidywania działań podejmowanych przez inne elementy systemu. W~przypadku agentów sztucznej inteligencji, algorytmy regułowe powielają zachowania prawdziwych graczy.

Z~racji na kooperacyjną konstrukcję Hanabi, sama emulacja typowych zachowań ludzkich graczy~nie wystarcza jednak, by~wygrać. Potrzebna jest także odpowiednia koordynacja działań pomiędzy agentami: strategia. Gracze muszą zwracać uwagę na~współuczestników rozgrywki i~na~bieżąco interpretować ich poczynania, by~móc wywnioskować, jaka seria ruchów doprowadzi do~najlepszej możliwej sytuacji. Prostym i~efektywnym sposobem na~zaimplementowanie strategii jest wyspecyfikowanie protokołu komunikacji pomiędzy agentami, który nadaje niektórym zagraniom dodatkowe znaczenie, rozumiane przez pozostałych graczy. Przykładowo, dobrym pomysłem może być zasada o~następującej treści: ``jeżeli inny gracz podpowiedział mi bez wyraźnej przyczyny jeden z kolorów, ujawniając w ten sposób kilka kart, najprawdopodobniej mogę je zagrać, w~kolejności od~lewej do~prawej''. Z~tego powodu implementacja agentów regułowych wiąże~się z~koniecznością bardzo dobrej znajomości zasad rządzących grą.

\section{Drzewa poszukiwań}

Istnieją dwa główne czynniki, które sprawiają, że~analiza stanu gry w~Hanabi jest trudnym zadaniem. Są~to: niepełna informacja o~aktualnym etapie rozgrywki, a~także losowość kart dobieranych z~talii. Udowodniono, że~nawet w uproszczonej wersji gry, w~której uczestnicy mogą patrzeć na~swoje karty, problem perfekcyjnego zagrania jest NP-kompletny\cite{NP-Complete}. Sprawia to, że~podejście do~zagadnienia w~sposób siłowy jest nieefektywne. Fakt ten, połączony z~trudnością opracowania funkcji oceniającej jakość zagrania, wyklucza użycie części możliwych rozwiązań problemu, takich jak algorytm alfa-beta.

Aby obejść trudność znalezienia perfekcyjnego zagrania, grupa Facebook Research zaproponowała rozwiązanie bazujące na~drzewie poszukiwań Monte Carlo\cite{MCTS}. Każdy z~graczy dysponuje zbiorem predefiniowanych akcji, które dostosowuje odpowiednio do~aktualnego stanu rozgrywki poprzez analizę prawdopodobieństwa zagrań, które mogą wykonać współuczestnicy. Algorytm bierze pod uwagę także szanse aktualnego gracza na~posiadanie w~ręce kart, które mogły zostać wylosowane z~talii. Aby przyspieszyć działanie programu, głębokość drzewa poszukiwań jest ograniczana, a agenci wykonują predefiniowane akcje i~nie~eksplorują nowych opcji, jeżeli wiązałoby~się to~z~przekroczeniem zadanego limitu obliczeń. 

Takie podejście pozwala na uzyskanie bardzo wysokich wyników, sięgających nawet 24.61~punktów w~rozgrywce dla dwóch graczy. Działanie algorytmu jest jednak kosztowne obliczeniowo, nawet przy znacznym ograniczeniu zakresu dokonywanych poszukiwań. Do~osiągnięcia tak wysokich punktacji potrzeba olbrzymich ilości obliczeń, które, z racji na możliwość ich zrównoleglenia, są~najczęściej dokonywane na~nowoczesnych kartach graficznych. Wyłączenie agentom możliwości dokonywania dodatkowych poszukiwań degeneruje ich do~postaci agentów regułowych, które, choć na~każdą akcję potrzebują zużycia istotnie mniejszej ilości zasobów, wciąż osiągają imponujący wynik 23~punktów.

\section{Algorytmy uczące}

Innym sposobem na~zaimplementowanie programu grającego w Hanabi jest użycie algorytmów uczących, które łączą zalety podejść regułowych i~poszukujących. Jak sugeruje nazwa, polegają one na~symulowaniu procesu akumulacji doświadczenia, podobnego do~tego doznawanego przez ludzkich graczy. W~toku ewolucji agent zdobywa wiedzę o~środowisku, w~którym operuje, dostosowując~się do~zmieniających warunków, wypracowując i~udoskonalając sposoby radzenia sobie w~zaprezentowanych sytuacjach. Tworzenie algorytmów uczących nie~wymaga ani szerokiej wiedzy o~zawiłościach zasad gry, ani kosztownych obliczeń, które byłyby wykonywane w~trakcie rozgrywki.

Wadą tego podejścia jest konieczność wyuczenia agenta odpowiednich zachowań. Odbywa~się to~poprzez zapewnienie mu~zestawu danych, na~których mógłby zdobyć doświadczenie. W~zależności od~uzyskiwanych wyników, decyzje algorytmu są~nagradzane lub karane. Dobranie odpowiednio różnorodnego zbioru uczącego, w~parze z funkcjami kwalifikującymi, pozwala programowi nie~tylko na~rozpoznawanie i~radzenie sobie z~najczęściej występującymi sytuacjami, ale i~generalizację zachowań, potrzebną do~wybrnięcia ze~stanów gry, które nie~zostały dotychczas napotkane.

Należy mieć na~uwadze, że~nieodpowiedni dobór zbioru uczącego lub funkcji, które oceniają poczynania agenta, potrafi doprowadzić do~anomalii w~procesie zdobywania wiedzy. Jeżeli zestawy testowe będą zbyt homogeniczne i~zbyt liczne w~stosunku do~osiągalnej liczby stanów rozgrywki, może dojść do~przeuczenia modelu, z~kolei zbyt krótka nauka nie~przygotowuje programu do~nietypowych sytuacji. Nieprawidłowości w~procedurach klasyfikujących, choć mniej zauważalne, także potrafią doprowadzić do~niepożądanych sytuacji, tak jak miało to~miejsce w~przypadku programu grającego w~produkcje na~platformę Nintendo Entertainment System. Agent ten, nie~chcąc doprowadzić do~przegranej, nauczył~się wstrzymywać rozgrywkę na~zawsze\cite{Mario}.

\section{Naginanie zasad gry}

W~oficjalnych zasadach gry nie~ma~wyszczególnionego przymusu udzielania podpowiedzi, które ujawniałyby jakiekolwiek karty. Jeżeli wybierzemy gracza, który~nie posiada żadnych czerwonych kart i~zdecydujemy się na~podpowiedzenie mu~czerwonego koloru, tura jest pomijana za~cenę żetonu podpowiedzi. Choć taki ruch wydaje~się~nie mieć sensu, gdyż podpowiedź można wykorzystać w~produktywny sposób, otwiera on~możliwość poważnego nagięcia zasad gry. Jeżeli każdej z~możliwych podpowiedzi przypiszemy unikatową wartość numeryczną, możemy za~ich pomocą przekazywać innym graczom informacje liczbowe. Jest to~powód, dla~którego możliwość udzielania pustych podpowiedzi jest uznawana w~społeczności graczy Hanabi za~kontrowersyjną, toteż w~niektórych edycjach gry została ona zakazana.

Fakt ten można wykorzystać do~stworzenia agenta, który za~pomocą pozornie bezwartościowych ruchów udziela podpowiedzi wszystkim graczom jednocześnie. Korzysta on~ze~słynnej zagadki logicznej, znanej jako problem więźniów i~kapeluszy, odpowiednio uogólnionej i~dopasowanej do~liczby graczy. Agent, który rozgrywa aktualną turę, oblicza idealne zagrania dla~innych uczestników rozgrywki, po~czym szyfruje je~do~postaci liczbowej. Kolejni gracze, znając podaną przez poprzednika wartość, po~rozpatrzeniu optymalnych zagrań innych graczy, są~w~stanie wywnioskować, jaki ruch powinni wykonać.

Według badań z 2017 roku\cite{HatPlayer}, agent ten uzyskuje maksymalną ilość punktów średnio w~92\%~rozgrywanych gier, co jest wynikiem bliskim optymalnemu. Jest to~imponujący rezultat, zarówno ze~względu na~bardzo szybkie działanie algorytmu, jak i~jego nadzwyczajną efektywność.

Niestety, taki sposób gry całkowicie zawodzi, gdy jeden z~graczy wyłamie się z~konwencji narzuconej przez protokół komunikacji. Dodatkowo, algorytm działa wyłącznie w~rozgrywce na~czterech oraz pięciu graczy, gdyż głównym powodem jego sukcesu jest możliwość przekazywania w~każdej z~podpowiedzi maksymalnej ilości informacji, toteż~zmniejszenie liczby graczy powoduje znaczne zredukowanie efektywności ruchów. Są~to~powody, dla których agent ten nie~nadaje~się do~rozgrywki z~człowiekiem.

\chapter{Ujęcie praktyczne}

W~celu zbadania zaprezentowanych sposobów tworzenia sztucznej inteligencji, która potrafiłaby grać w~Hanabi, zaimplementowaliśmy ośmiu agentów. Aby umożliwić testowanie ich możliwości, opracowaliśmy także graficzny interfejs (patrz: \hyperref[fig:gui]{Rysunek 2.1}), pozwalający użytkownikowi na~aktywne uczestnictwo w~rozgrywce, złożonej z~dowolnie dobranego przez niego składu graczy komputerowych.

Siedmiu agentów jest opartych o~systemy regułowe, różniące~się stopniem zaawansowania wdrażanych strategii. Aby usprawnić działanie jednego z~agentów regułowych, użyliśmy bayesowskiej metody optymalizacji hiperparametrów, korzystającej z~procesów gaussowskich. Ostatni z~zaprezentowanych agentów stosuje algorytm uczący ze~wzmocnieniem, oparty o~funkcje heurystyczne generalizujące stan gry, wspomagane procesami decyzyjnymi Monte Carlo.

Systemy zasad zostały oparte o~nasze własne spostrzeżenia, których doszukaliśmy~się poprzez obserwację rozgrywek ludzkich graczy i~uznaliśmy za~kluczowe do~gry na~wysokim poziomie. Każdy kolejny agent regułowy bazuje na~konceptach, które zostały użyte do~zbudowania jego poprzedników, i~usprawnia je. Takie podejście~prowadzi do~naturalnej progresji prezentowanych warstw abstrakcji.

\section{Cheater}

Cheater jest agentem, którego strategia polega na~bezmyślnym oszukiwaniu. Program zawsze posiada komplet informacji o~kartach, którymi dysponuje, lecz~nie wykorzystuje dostępnej mu~wiedzy w~optymalny sposób: zagrywa karty, gdy to~możliwe, w~przeciwnym wypadku losowo je~odrzuca.

\subsection*{Nowe pojęcia}

\begin{itemize}
	\item Ujawniona karta:
	
	Jeżeli gracz jest w posiadaniu karty, o~której posiada komplet informacji, kartę tę~określimy jako ujawnioną.
	
	\newpage %
	\item Bezpieczna akcja:
	
	Jeżeli gracz może wykonać ruch, który nie doprowadzi do~utraty żetonów życia, taką akcję określimy jako bezpieczną.
	
	\item Bezpieczne zagranie:
	
	Jeżeli gracz posiada ujawnioną kartę, którą może bezpiecznie (czyli poprawnie) położyć na~planszy, wykonanie takiej akcji nazwiemy bezpiecznym zagraniem.
\end{itemize}

\subsection*{Schemat działania}

\begin{enumerate}
	\item Podejrzyj posiadane karty.
	\item Jeżeli posiadasz bezpieczne zagranie, wykonaj je.
	\item Odrzuć losową kartę.
\end{enumerate}

\subsection*{Osiągi}

\begin{table}[h!]
	\centering
	\begin{tabular}{ r|c|c|c|c| }
		\multicolumn{1}{r}{}
		 & \multicolumn{1}{c}{2 graczy}
		 & \multicolumn{1}{c}{3 graczy}
		 & \multicolumn{1}{c}{4 graczy}
		 & \multicolumn{1}{c}{5 graczy} \\
		\cline{2-5}
		średnia & 20.61 & 22.33 & 21.76 & 22.03 \\
		\cline{2-5}
		mediana & 21 & 23 & 22 & 22 \\
		\cline{2-5}
		wariancja & 6.49 & 4.01 & 4.45 & 3.51 \\
		\cline{2-5}
		odchylenie standardowe & 2.55 & 2.00 & 2.11 & 1.87 \\
		\cline{2-5}
	\end{tabular}
	\caption{Wyniki agenta Cheater (rozmiar próby: 1000 gier)}
	\label{table:Cheater}
\end{table}

Uzyskane wyniki są~bardzo dobre, lecz cechują~się stosunkowo wysoką wariancją. Agenci są~skłonni do~odrzucania kluczowych kart i~nie~zwracają uwagi na~fakt, że~może to~znacząco utrudnić lub nawet uniemożliwić dalsze zdobywanie punktów.

\section{SmartCheater}

SmartCheater dodaje głębię poczynaniom agenta Cheater. Program zawsze posiada komplet informacji o~kartach, którymi dysponuje, i~wykorzystuje tę~wiedzę w~sposób bliski optymalnemu, stosując strategię przedłużającą rozgrywkę. Agent potrafi oceniać przydatność kart i~za~wszelką cenę unika ruchów, które utrudniają lub uniemożliwiają wygraną.

\subsection*{Nowe pojęcia}

\begin{itemize}
	\item Zagranie lawinowe:
	
	Jeżeli położenie karty na~planszy pozwala innym graczom na~wykonywanie nowych ruchów, które kontynuują dokładanie kart do~stosów, takie zagranie nazwiemy lawinowym. Przykładowo: jeżeli aktualny gracz poprawnie zagra czerwoną kartę o~numerze~1, zaś dwaj kolejni gracze posiadają kolejno czerwone~karty o~numerach~2~i~3, zagranie jest lawinowe.
	
	\item Bezużyteczna karta:
	
	Jeżeli nie~istnieje możliwość poprawnego zagrania karty, jest ona bezużyteczna i~może być odrzucona bez żadnych konsekwencji. Przykładowo: gdy stos kart koloru zielonego jest w~pełni ułożony, nie~można poprawnie zagrać żadnych dodatkowych kart koloru zielonego.
	
	\item Karta krytyczna:
	
	Jeżeli karta nie~jest bezużyteczna, a~jej odrzucenie wiąże się z~niemożnością zagrania przez któregokolwiek z~graczy innej jej kopii, karta jest krytyczna. Przykładowo: każda karta o~numerze~5 jest krytyczna.
	
	\item Spasowanie tury:
	
	Jeżeli na planszy znajduje~się co~najmniej jeden żeton podpowiedzi, udzielenie losowej podpowiedzi dowolnemu z~graczy jest równoznaczne ze~spasowaniem tury.
\end{itemize}

\subsection*{Schemat działania}

\begin{enumerate}
	\item Podejrzyj posiadane karty.
	\item Jeżeli posiadasz bezpieczne zagranie, wykonaj je. Pierwszeństwo nadawane jest zagraniom lawinowym, zaś w~razie konfliktu wygrywa losowa karta o~najniższym numerze.
	\item Jeżeli możesz spasować turę, a~ruch ten~nie~przeszkodzi żadnemu z~pozostałych graczy w~wykonaniu jednej z~akcji od~2~do~5, zrób to.
	\item Jeżeli posiadasz bezużyteczną kartę, odrzuć ją.
	\item Jeżeli jest to~możliwe, spasuj turę.
	\item Jeżeli posiadasz kartę, której kopia znajduje się w~ręce innego gracza, odrzuć ją.
	\item Jeżeli posiadasz kartę, która nie jest krytyczna, odrzuć ją.
	\item Odrzuć losową kartę krytyczną o~najwyższym numerze.
\end{enumerate}

\subsection*{Osiągi}

\begin{table}[h!]
	\centering
	\begin{tabular}{ r|c|c|c|c| }
		\multicolumn{1}{r}{}
		 & \multicolumn{1}{c}{2 graczy}
		 & \multicolumn{1}{c}{3 graczy}
		 & \multicolumn{1}{c}{4 graczy}
		 & \multicolumn{1}{c}{5 graczy} \\
		\cline{2-5}
		średnia & 24.72 & 24.93 & 24.87 & 24.86 \\
		\cline{2-5}
		mediana & 25 & 25 & 25 & 25 \\
		\cline{2-5}
		wariancja & 0.65 & 0.11 & 0.19 & 0.19 \\
		\cline{2-5}
		odchylenie standardowe & 0.80 & 0.33 & 0.44 & 0.44 \\
		\cline{2-5}
	\end{tabular}
	\caption{Wyniki agenta SmartCheater (rozmiar próby: 1000 gier)}
	\label{table:SmartCheater}
\end{table}

Jak można oczekiwać po~agencie, który posuwa~się do~oszustwa bliskiego doskonałemu, jego wyniki są~bardzo wysokie, nawet w~przypadku partii na~dwóch graczy, które cechują~się największymi dysproporcjami w~rozgrywce.

\section{StoppedClock}

Agent StoppedClock wykonuje losowe ruchy, które nie~mogą doprowadzić do~utraty żetonów życia. Jego postępowanie, choć bardzo chaotyczne, czasem prowadzi do~wysokich punktacji, zgodnie z~maksymą: ``nawet zepsuty zegar dwa razy na~dobę pokazuje właściwą godzinę''.

\subsection*{Schemat działania}

Ruch agenta polega na~zbadaniu, które z~trzech dostępnych akcji mogą być wykonane, a~następnie wylosowaniu i~wykonaniu jednej z~nich. Ponieważ tylko druga i~trzecia akcja mogą wzajemnie~się wykluczać, gracz zawsze będzie w~stanie wykonać co~najmniej jeden z~ruchów. Dostępne akcje to:

\begin{itemize}
	\item Jeżeli posiadasz bezpieczne zagranie, wykonaj je.
	\item Odrzuć losową kartę, priorytetyzując tę, która ma najmniej ujawnionych informacji. Akcja nie~może być wykonana, jeżeli na~planszy znajduje~się osiem żetonów podpowiedzi.
	\item Wybierz losowego gracza, który posiada co~najmniej jedną kartę o~niepełnej informacji, wylosuj jedną, po~czym losowo podpowiedz kolor lub numer wybranej karty, jeżeli nie~zostały one uprzednio ujawnione. Jeżeli wszyscy gracze posiadają komplet informacji, spasuj turę.
\end{itemize}

\subsection*{Osiągi}

\begin{table}[h!]
	\centering
	\begin{tabular}{ r|c|c|c|c| }
		\multicolumn{1}{r}{}
		 & \multicolumn{1}{c}{2 graczy}
		 & \multicolumn{1}{c}{3 graczy}
		 & \multicolumn{1}{c}{4 graczy}
		 & \multicolumn{1}{c}{5 graczy} \\
		\cline{2-5}
		średnia & 5.95 & 6.34 & 5.46 & 4.42 \\
		\cline{2-5}
		mediana & 6 & 6 & 5 & 4 \\
		\cline{2-5}
		wariancja & 7.08 & 5.16 & 3.86 & 3.02 \\
		\cline{2-5}
		odchylenie standardowe & 2.66 & 2.27 & 1.96 & 1.74 \\
		\cline{2-5}
	\end{tabular}
	\caption{Wyniki agenta StoppedClock (rozmiar próby: 1000 gier)}
	\label{table:StoppedClock}
\end{table}

Agent wykonujący wyłącznie losowe ruchy jest pierwszym, który nie~oszukuje. Niestety, ma~on~nikłą szanse na~uzyskanie wysokich wyników, a~jego konstrukcja nigdy nie~pozwoli na~wygraną. Jest to~spowodowane ograniczeniem, które umożliwia programowi funkcjonowanie: wykonywane ruchy muszą być bezpieczne, co~koliduje z~ograniczoną liczbą żetonów podpowiedzi.

\section{SimpleDistrustful}

Agent SimpleDistrustful imituje postępowanie człowieka, który po~raz pierwszy gra w~Hanabi: wykonuje wyłącznie bezpieczne ruchy, lecz robi to~umiejętnie, gdyż potrafi przeprowadzać pewne wnioskowania, które na~jego miejscu mogłyby dokonać osoby o~podstawowej znajomości zasad. Program nie~udziela redundantnych podpowiedzi, stroni od~odrzucania kart, które mogą~się jeszcze przydać, a~także rozumie, że~niektóre karty zawsze można bezpiecznie zagrać lub odrzucić -~mimo posiadania o~nich niepełnej informacji.

\subsection*{Nowe pojęcia}

\begin{itemize}
	\item Inferowalna karta:

	Użyteczności niektórych kart można dociec poprzez analizę aktualnego stanu rozgrywki. Przykładowo: na~początku gry zawsze można zagrać dowolną kartę z~numerem~1, niezależnie od~tego, czy jej kolor został ujawniony. Podobnie, jeżeli nie~odrzucono ani nie zagrano żadnej karty o~numerze~4, żadna taka karta nie~może być krytyczna. Karty o~takiej właściwości nazywamy inferowalnymi, czyli dającymi~się wywnioskować. Karty, które są~ujawnione, są~trywialnie inferowalne.
	
	\item Wartościowa podpowiedź:
	
	Zdarzają się sytuacje, w~których udzielanie pewnych informacji jest niepożądane. Przykładowo: jeżeli dwóch lub więcej uczestników rozgrywki posiada kopię danej karty, warto skupić~się ujawnianiu wyłącznie jednej z~nich. Wartościowa podpowiedź to~taka, która jest lokalnie optymalna, gdyż unika powtarzania znanych informacji.
	
	\item Najbliższy gracz:
	
	Uczestnik rozgrywki, który po~zakończeniu akcji aktualnego gracza wykona ruch najwcześniej.
	
\end{itemize}

\subsection*{Schemat działania}

\begin{enumerate}
	\item Jeżeli posiadasz inferowalnie bezpieczne zagranie, wykonaj je.
	\item Jeżeli jest to~możliwe, wybierz najbliższego gracza, któremu możesz udzielić wartościowej podpowiedzi, po~czym zrób to, nadając priorytet kartom o~niższych numerach.
	\item Jeżeli posiadasz nieujawnioną kartę, odrzuć ją.
	\item Jeżeli posiadasz kartę, która nie jest inferowalnie krytyczna, odrzuć ją.
	\item Odrzuć losową kartę.
\end{enumerate}

\subsection*{Osiągi}

\begin{table}[h!]
	\centering
	\begin{tabular}{ r|c|c|c|c| }
		\multicolumn{1}{r}{}
		 & \multicolumn{1}{c}{2 graczy}
		 & \multicolumn{1}{c}{3 graczy}
		 & \multicolumn{1}{c}{4 graczy}
		 & \multicolumn{1}{c}{5 graczy} \\
		\cline{2-5}
		średnia & 13.29 & 15.51 & 15.05 & 13.99 \\
		\cline{2-5}
		mediana & 14 & 16 & 15 & 14 \\
		\cline{2-5}
		wariancja & 9.54 & 1.61 & 1.27 & 0.93 \\
		\cline{2-5}
		odchylenie standardowe & 3.09 & 1.27 & 1.13 & 0.96 \\
		\cline{2-5}
	\end{tabular}
	\caption{Wyniki agenta SimpleDistrustful (rozmiar próby: 1000 gier)}
	\label{table:SimpleDistrustful}
\end{table}

Agent nie~uzyskuje wysokich wyników, gdyż opiera swoje działania na~przeciętnej strategii. Udzielanie wyłącznie wartościowych informacji skłania graczy do~zagrywania kart, lecz często nie~wystarcza do~ochrony kart krytycznych -~nieujawnione karty wymagają pewnej formy ochrony, gdyż mogły być dobrane dopiero niedawno.

\section{Distrustful}

Agent Distrustful dopracowuje strategię stosowaną przez SimpleDistrustful. Program nadal gra zachowawczo i~nie~bierze pod uwagę możliwości udzielania ani otrzymywania implicytnych informacji, w~zamian polegając wyłącznie na~maksymalnie rozszerzonym systemie dedukcji. Dodatkowo, aby zwiększyć potencjał każdej z~akcji, agent dynamicznie dostosowuje swój sposób postępowania do~aktualnego stanu rozgrywki i~zmienia priorytety manewrów, w~zależności od~ilości pozostałych żetonów podpowiedzi.

\subsection*{Nowe pojęcia}

\begin{itemize}
	\item Heurystycznie wartościowa akcja:
	
	Niektóre akcje, takie jak bezpieczne zagranie, są~w~oczywisty sposób korzystne dla postępu gry. Niestety, przypisanie części ruchów jednoznacznej wartości liczbowej, wskazującej na jej przydatność w~kontekście danego momentu rozgrywki, jest trudnym zadaniem, gdyż pełna analiza możliwych następstw akcji wymagałaby wysokiego nakładu mocy obliczeniowej. Aby uniknąć tego problemu, można zastosować funkcje heurystyczne, które przybliżają wartości poszukiwanych~ocen. Czasami jednak nie~istnieje ruch, który byłby dobry, dlatego każda otrzymana wartość musi także być wyższa od~pewnego ustalonego pułapu granicznego. Akcja, która uzyska najwyższą notę, a~także przejdzie test jakości, jest nazywana heurystycznie optymalną.
	
	\item Najstarsza karta:
	
	Karta, która znajduje~się w~ręce danego gracza od~największej ilości tur. W~razie konfliktu za~najstarszą kartę uznawana jest ta~wysunięta najbardziej na~lewo.

\end{itemize}

\subsection*{Schemat działania}

Agent jest złożony z~siedmiu modułów, które są~rozważane po~kolei w~jednej z~trzech konfiguracji. Każdy moduł, poza ostatnim, może zawieść. Ich kolejność zależy od~ilości posiadanych żetonów podpowiedzi:

\begin{itemize}
	\item Mniej niż 3~żetony: sekwencja 1, 2, 5, 3, 4, 7 (moduł 6 nie jest używany).
	\item Od~3~do~7 żetonów: sekwencja 1, 2, 3, 4, 5, 6, 7.
	\item Dokładnie 8~żetonów: sekwencja 1, 2, 3, 4, 6, 5, 7.
\end{itemize}
Używane moduły to:

\begin{enumerate}
	\item ``Necessary tip'':
	
	Jeżeli najbliższy gracz planuje w~kolejnym ruchu odrzucić kartę krytyczną, powstrzymaj go~poprzez udzielenie odpowiedniej podpowiedzi.

	\item ``Obvious play'':
	
	Jeżeli posiadasz inferowalnie bezpieczne zagranie, wykonaj~je, nadając priorytet kartom o~niższych numerach.

	\item ``Good play tip'':
	
	Jeżeli posiadasz co~najmniej dwa żetony podpowiedzi i~istnieje heurystycznie wartościowa podpowiedź, która umożliwi innemu graczowi inferowalnie bezpieczne zagranie, udziel jej.

	\item ``Discard tip'':
	
	Jeżeli posiadasz co~najmniej dwa żetony podpowiedzi i~istnieje heurystycznie wartościowa podpowiedź, która umożliwi innemu graczowi odrzucenie inferowalnie bezużytecznej karty, udziel jej.

	\item ``Obvious discard'':
	
	Jeżeli posiadasz inferowalnie bezużyteczną kartę, odrzuć ją.

	\item ``Mediocre play tip'':
	
	Jeżeli posiadasz co~najmniej dwa żetony podpowiedzi, spróbuj wykonać akcję modułu numer~3 (``Good play tip''), ale ze~znacznie niższym pułapem jakościowym.

	\item ``Guess discard'':
	\begin{enumerate}
		\item Jeżeli posiadasz nieujawnione karty, odrzuć najstarszą z~nich.
		\item Jeżeli posiadasz karty, które nie~są~inferowalnie krytyczne, odrzuć najstarszą z~nich.
		\item Odrzuć najstarszą kartę.
	\end{enumerate}

\end{enumerate}

Zmiana kolejności manewrów pozwala na~kontrolowanie priorytetów agenta: im~mniej żetonów podpowiedzi znajduje~się w~grze, tym bardziej nagląca staje~się potrzeba ich odzyskania. Nigdy~nie wiadomo, czy i~kiedy któryś z~graczy dobierze kartę wymagającą natychmiastowej interwencji w~formie dodatkowych informacji. Jest to~powód, dla którego moduły udzielające podpowiedzi do~kart niebędących krytycznymi zawodzą przy próbie wykorzystania ostatniego z~żetonów.

Wagi funkcji heurystycznych zostały dobrane eksperymentalnie. Poza sprawdzaniem, czy dana podpowiedź jest wartościowa, na~werdykt algorytmu wpływają między innymi następujące czynniki:
\begin{itemize}
	\item stopień ujawnienia karty
	\item ilość kart, które muszą być zagrane, zanim karta będzie bezpieczna
	\item odległość właściciela karty od~aktualnego gracza
	\item numer karty
	\item wagi kart, które zostaną ujawnione wraz z~rozważaną kartą
\end{itemize}

Początkowo, moduł numer~1 (``Necessary tip'') przyjmował inną formę. Zamiast ostrzegać graczy przed odrzucaniem krytycznych kart natychmiast po~wykryciu zagrożenia, udzielenie kluczowej informacji było odwlekane aż~do~tury gracza, który jako ostatni mógł zapobiec nieszczęściu. W~praktyce doprowadzało to~do~powstawania cykli, w~których gracze naprzemiennie udzielali podpowiedzi i~odrzucali karty, przez co~częstotliwość zagrań drastycznie spadała. Choć pierwotny moduł znacznie pogarszał wyniki agenta Distrustful, jego zrewidowana wersja stanowi serce kolejnego, znacznie bardziej zaawansowanego agenta.

\subsection*{Osiągi}

\begin{table}[H]
	\centering
	\begin{tabular}{ r|c|c|c|c| }
		\multicolumn{1}{r}{}
		 & \multicolumn{1}{c}{2 graczy}
		 & \multicolumn{1}{c}{3 graczy}
		 & \multicolumn{1}{c}{4 graczy}
		 & \multicolumn{1}{c}{5 graczy} \\
		\cline{2-5}
		średnia & 17.46 & 16.45 & 15.75 & 14.48 \\
		\cline{2-5}
		mediana & 18 & 16 & 16 & 14 \\
		\cline{2-5}
		wariancja & 2.59 & 1.26 & 1.06 & 0.84 \\
		\cline{2-5}
		odchylenie standardowe & 1.61 & 1.12 & 1.03 & 0.92 \\
		\cline{2-5}
	\end{tabular}
	\caption{Wyniki agenta Distrustful (rozmiar próby: 1000 gier)}
	\label{table:Distrustful}
\end{table}

Mimo znacznego wzmocnienia strategii agenta SimpleDistrustful, osiągane wyniki nadal nie~zachwycają. Gracze udzielają heurystycznie wartościowych podpowiedzi, chronią krytyczne karty poprzez system wieku i~oszczędzają żetony na~czarną godzinę -~trudno jednak doszukiwać~się między nimi głębszej współpracy, która jest kluczowa dla wysokich osiągów. Zagrywane są~wyłącznie karty inferowalnie bezpieczne, co, nawet przy bardzo ostrożnej i~wydajnej komunikacji, nie~wystarcza do~podjęcia równej walki z~szybko kurczącą~się ilością żetonów podpowiedzi. Aby przekroczyć wyczekiwany próg 20~punktów, potrzeba strategii, która dopuszcza pewne ryzyko.

\section{Trustful}

WIP

\subsection*{Osiągi}

\begin{table}[h!]
	\centering
	\begin{tabular}{ r|c|c|c|c| }
		\multicolumn{1}{r}{}
		 & \multicolumn{1}{c}{2 graczy}
		 & \multicolumn{1}{c}{3 graczy}
		 & \multicolumn{1}{c}{4 graczy}
		 & \multicolumn{1}{c}{5 graczy} \\
		\cline{2-5}
		średnia & 18.11 & 20.50 & 20.45 & 19.16 \\
		\cline{2-5}
		mediana & 19 & 21 & 20 & 19 \\
		\cline{2-5}
		wariancja & 10.91 & 2.65 & 2.71 & 2.64 \\
		\cline{2-5}
		odchylenie standardowe & 3.30 & 1.63 & 1.65 & 1.63 \\
		\cline{2-5}
	\end{tabular}
	\caption{Wyniki agenta Trustful (rozmiar próby: 1000 gier)}
	\label{table:Trustful}
\end{table}


\section{BayesianTrustful}

WIP

\section{Reinforced}

WIP

\chapter{Wnioski}

WIP

\begin{thebibliography}{4}

\bibitem{NP-Complete} J.-F Baffier i in., \textit{Hanabi is NP-complete, Even for Cheaters who Look at Their Cards}, 2017. URL: 
\href{https://arxiv.org/pdf/1603.01911.pdf}{\textbf{link}} (term. wiz. 11.01.2020)

\bibitem{MCTS} A. Lerer, H. Hu, J. Foerster, N. Brown, \textit{Building AI that can master complex cooperative games with hidden information}, 2019. URL: 
\href{https://ai.facebook.com/blog/building-ai-that-can-master-complex-cooperative-games-with-hidden-information/}{\textbf{link}} (term. wiz. 11.01.2020)

\bibitem{Mario} T. Murphy, \textit{The First Level of Super Mario Bros. is Easy with Lexicographic Orderings and Time Travel . . . after that it gets a little tricky.}, 2013. URL: 
\href{http://www.cs.cmu.edu/~tom7/mario/mario.pdf}{\textbf{link}} (term. wiz. 11.01.2020)

\bibitem{HatPlayer} B. Bouzy, \textit{Playing Hanabi Near-Optimally}, 2017. URL: 
\href{http://helios.mi.parisdescartes.fr/~bouzy/publications/bouzy-hanabi-2017.pdf}{\textbf{link}} (term. wiz. 11.01.2020)

\end{thebibliography}

\end{document}
